BACHELOR'S THESIS | ABSTRACT

TURKU UNIVERSITY OF APPLIED SCIENCES

ELECTRONIC ENGINEERING

2017 | \pageref{LastPage} PAGES


\vspace{10mm}
{\Large Jarmo Kivekäs \par}

\vspace{10mm}
{\huge Spectrum Monitoring System Implementation Using Software-Defined Radio \par}

\vspace{10mm}



This thesis covers general theory about the applications radio spectrum occupancy
monitoring, and the methods behind it, and describes essential elements of a
spectrum monitoring system that was implemented as part of the work for this
thesis using a National Instruments Universal Radio Peripheral and custom
application logic. The intention of the work was not to implement a feature-rich
monitoring system but to instead to use the implementation process of a simpler
system as a way to create a solid base understanding of the various parts of such
a system by finding solutions to problems encountered on the way.
Measurements were done to observe the behavior of phenomena such as CIC roll-off,
DC-offset, and the trade-off relations between system characteristics such as
frequency resolution, temporal resolution, required computational power,
and data set size.



\vspace{30mm}

KEYWORDS:

specturm monitoring, software-defined radio, dynamic spectrum access, digital signal processing

%%%%%%%%%%%%%%%%%%%%%%%%%%%%%%%%%%%%%%%%%%
%%%%%%%%%%%%%%%%%%%%%%%%%%%%%%%%%%%%%%%%%%
%%%%%%%%5 basically repeat everything verbatim in finnish
%%%%%%%%%%%%%%%%%%%%%%%%%%%5


\clearpage


OPINNÄYTETYÄ (AMK) | TIIVISTELMÄ

TURUN AMMATTIKORKEAKOULU

ELEKTRONIIKKA INSINÖÖRI

2017 | \pageref{LastPage} SIVUA


\vspace{10mm}
{\Large Jarmo Kivekäs \par}

\vspace{10mm}
{\huge Ohjelmistoradioon pohjautuvan spektrinmonitorointijärjestelmän toteutus \par}

\vspace{10mm}

Tämä opinnäytetyö esitää radiospektin monitoroinintiin liittyvää teoriaa
ja kuvailee National Instruments Universal Software Radio Peripheral -ohjelmitoradiolla
toteuteun spektinmonitorointijärjestelmän keskeisiä osia ja toimintoja.
Työn tarkoituksena ei ollut toteutaa kattavaa ja monipuolista monitorointijärejstelmää,
enneminkin tarkoituksena oli luoda syvempi ymmärrys käytettävien komponenttion
tomintaperiaatteista yksinkertaisemman järjestelmän toteutusprosessia apuna käyttäen.
Toteutuksen aikana törmättiin havaittiin tomintaan vaikuttavia ilmiöitä joiden negatiivinen
vaikutus järjestelmään minimoitiin tutkimalla ja ymmärtämällä niiden käyttäytmisen.

\vspace{30mm}

AVAINSANAT:

spektrin monitorointi, ohjelmistoradio, digitaalinen signaalinkäsittely
