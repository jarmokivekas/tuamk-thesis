BACHELOR'S THESIS | ABSTRACT

TURKU UNIVERSITY OF APPLIED SCIENCES

Electronics

2017 | \pageref{LastPage}


\vspace{7mm}
{\Large Jarmo Kivekäs \par}

\vspace{7mm}
{\huge SPECTRUM MONITORING SYSTEM IMPLEMENTATION USING SOFTWARE-DEFINED RADIO \par}

\vspace{7mm}



This thesis covers general theory about the applications of radio spectrum occupancy
monitoring, and the methods behind it, and describes essential elements of a
spectrum monitoring system that was implemented as part of the work for this
thesis using a National Instruments Universal Radio Peripheral and custom
application logic. The intention of the work was not to implement a feature-rich
monitoring system but to instead to use the implementation process of a simpler
system as a way to create a solid understanding of the parts of such
a system by finding solutions to problems encountered on the way.
Measurements were done to observe the behavior of phenomena such as CIC roll-off,
DC-offset, and the trade-off relations between system characteristics such as
frequency resolution, temporal resolution, required computational power,
and data set size.



\vspace{25mm}

KEYWORDS:

spectrum monitoring, software-defined radio, digital signal processing

%%%%%%%%%%%%%%%%%%%%%%%%%%%%%%%%%%%%%%%%%%
%%%%%%%%%%%%%%%%%%%%%%%%%%%%%%%%%%%%%%%%%%
%%%%%%%%5 basically repeat everything verbatim in finnish
%%%%%%%%%%%%%%%%%%%%%%%%%%%5


\clearpage


OPINNÄYTETYÖ (AMK) | TIIVISTELMÄ

TURUN AMMATTIKORKEAKOULU

Elektroniikka

2017 | \pageref{LastPage}


\vspace{7mm}
{\Large Jarmo Kivekäs \par}

\vspace{7mm}
{\huge SPEKTRINMONITOROINTIJÄRJESTELMÄN TOTEUTUS OHJELMISTORADIOLLA \par}

\vspace{7mm}


Tämä opinnäytetyö käsittelee radiospektrin monitorointiin liittyvää teoriaa ja kuvailee National Instruments Universal Software Radio Peripheral -ohjelmistoradiolla (USRP) toteutetun spektrinmonitorointijärjestelmän keskeisiä osia ja toimintoja.
Työn tarkoituksena oli luoda syvempi ymmärrys tarvittavien komponenttien toiminnasta ja niiden aiheuttamista ongelmista käyttäen järjestelmän toteutusprosessia tukena ongelmien löytämistä varten.

Toteutuksen aikana havaittiin toimintaan vaikuttavia ilmiöitä, joita tutkittiin jotta niiden negatiivinen vaikutus voitiin minimoida.
Monitorointijärjestelmän logiikka toteutettiin käyttämällä GNURadio -koodikirjastoa signaalinkäsittelyyn sekä USRP:n ohjaamista varten.
Python -skriptikieltä käytettiin korkeamman tason toimintojen toteuttamisessa.
Järjestelmällä kerättyä mittaustietoa verrattiin perinteisen spektrianalysaattorin mittatietoihin, ja todettiin tietojen täsmäävän toistensa kanssa.
Työssä verrataan mittaustulosten esittelyä histogrammina, jolloin informaatio tehon jakautumisesta säilyy, toisin kuin tavanomaisessa tehospektrissä.

Työssä selvitettiin miten ohjelmistoradioiden etuosan epäideaalisuudet aiheuttavat spektrissä näkyviä tasajännitepiikkejä ja kuinka sopimattoman näytetaajuuden valitseminen voi aiheuttaa suuria vääristymiä mitatun spektrin amplitudissa vastaanottimen jälleennäytteistämisvaiheen jälkeen.
Ohjelmistoradiota käyttäessä, voidaan spektrin mittauksessa tehdä kompromisseja tarvittavan levytilan, laskentakyvyn, taajuusresoluution ja aikaresoluution välillä, riippuen siitä millaisia ilmiöitä halutaan mitata.
Lopuksi pohditaan, miten etuosan toimintaan voidaan vaikuttaa laitteen suojelemiseksi suurtehosilta signaaleilta, ja miten etuosan aiheuttamia IQ-näytteistämisen epätasapainoisuuksia korjataan empiirisen kalibroinnin avulla.




\vspace{10mm}

AVAINSANAT:

spektrin monitorointi, ohjelmistoradio, digitaalinen signaalinkäsittely
